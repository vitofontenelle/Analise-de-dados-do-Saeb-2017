%% abtex2-modelo-glossarios.tex, v-1.9.7 laurocesar
%% Copyright 2012-2018 by abnTeX2 group at http://www.abntex.net.br/
%%
%% This work may be distributed and/or modified under the
%% conditions of the LaTeX Project Public License, either version 1.3
%% of this license or (at your option) any later version.
%% The latest version of this license is in
%%   http://www.latex-project.org/lppl.txt
%% and version 1.3 or later is part of all distributions of LaTeX
%% version 2005/12/01 or later.
%%
%% This work has the LPPL maintenance status `maintained'.
%%
%% The Current Maintainer of this work is the abnTeX2 team, led
%% by Lauro César Araujo. Further information are available on
%% http://www.abntex.net.br/
%%
%% This work consists of the files abtex2-modelo-glossarios.tex,
%% abntex2-modelo-include-comandos and abntex2-modelo-references.bib
%%
                     
% ------------------------------------------------------------------------
% ------------------------------------------------------------------------
% abnTeX2: Exemplo de glossários com o pacote glossaries e abntex2
% ------------------------------------------------------------------------
% ------------------------------------------------------------------------
 
\documentclass[
	% -- opções da classe memoir --
	12pt,				% tamanho da fonte
	openright,			% capítulos começam em pág ímpar (insere página vazia caso preciso)
	twoside,			% para impressão em recto e verso. Oposto a oneside
	a4paper,			% tamanho do papel. 
	% -- opções da classe abntex2 --
	%chapter=TITLE,		% títulos de capítulos convertidos em letras maiúsculas
	%section=TITLE,		% títulos de seções convertidos em letras maiúsculas
	%subsection=TITLE,	% títulos de subseções convertidos em letras maiúsculas
	%subsubsection=TITLE,% títulos de subsubseções convertidos em letras maiúsculas
	% -- opções do pacote babel --
	english,			% idioma adicional para hifenização
	french,				% idioma adicional para hifenização
	spanish,			% idioma adicional para hifenização
	brazil,				% o último idioma é o principal do documento
	]{abntex2}
 
 
% ---
% PACOTES
% ---
 
% ---
% Pacotes fundamentais
% ---
\usepackage{lmodern}			% Usa a fonte Latin Modern			
\usepackage[T1]{fontenc}		% Selecao de codigos de fonte.
\usepackage[utf8]{inputenc}		% Codificacao do documento (conversão automática dos acentos)
\usepackage{indentfirst}		% Indenta o primeiro parágrafo de cada seção.
\usepackage{color}				% Controle das cores
\usepackage{graphicx}			% Inclusão de gráficos
\usepackage{microtype} 			% para melhorias de justificação
\usepackage[normalem]{ulem}
\usepackage{tablefootnote}

% ---

% ---
% Pacotes glossaries
% ---
\usepackage[subentrycounter,seeautonumberlist,nonumberlist=true]{glossaries}
% para usar o xindy ao invés do makeindex:
%\usepackage[xindy={language=portuguese},subentrycounter,seeautonumberlist,nonumberlist=true]{glossaries}
% ---

% ---
% Pacotes de citações
% ---
\usepackage[brazilian,hyperpageref]{backref}	 % Paginas com as citações na bibl
\usepackage[alf]{abntex2cite}	% Citações padrão ABNT
 
% ---
% Informações de dados para CAPA e FOLHA DE ROSTO
% ---
\titulo{Análise Descritiva dos Dados do Saeb - 2017}
\autor{Vito Genovese F. N. de Amorim}
\local{Brasil}
\data{10/11/2024}
\orientador{Lauro César Araujo}
\coorientador{Equipe \abnTeX}
\instituicao{%
  Universidade do Brasil -- UBr
  \par
  Faculdade de Arquitetura da Informação
  \par
  Programa de Pós-Graduação}
\tipotrabalho{Tese (Doutorado)}
% O preambulo deve conter o tipo do trabalho, o objetivo,
% o nome da instituição e a área de concentração
\preambulo{Modelo de texto com glossário conforme as
as normas ABNT apresentado à comunidade de usuários \LaTeX.}
% ---
 
 
% ---
% Configurações de aparência do PDF final
 
% alterando o aspecto da cor azul
\definecolor{blue}{RGB}{41,5,195}
 
% informações do PDF
\makeatletter
\hypersetup{
     	%pagebackref=true,
		pdftitle={\@title}, 
		pdfauthor={\@author},
    	pdfsubject={\imprimirpreambulo},
	    pdfcreator={LaTeX with abnTeX2},
		pdfkeywords={abnt}{latex}{abntex}{abntex2}{glossários}, 
		colorlinks=true,       		% false: boxed links; true: colored links
    	linkcolor=blue,          	% color of internal links
    	citecolor=blue,        		% color of links to bibliography
    	filecolor=magenta,      		% color of file links
		urlcolor=blue,
		bookmarksdepth=4
}
\makeatother

% ---
% Tamanho das notas de roda pé
% ---
\makeatletter
\renewcommand\footnotesize{%
   \@setfontsize\footnotesize\@ixpt{11}%
   \abovedisplayskip 8\p@ \@plus2\p@ \@minus4\p@
   \abovedisplayshortskip \z@ \@plus\p@
   \belowdisplayshortskip 4\p@ \@plus2\p@ \@minus2\p@
   \def\@listi{\leftmargin\leftmargini
               \topsep 4\p@ \@plus2\p@ \@minus2\p@
               \parsep 2\p@ \@plus\p@ \@minus\p@
               \itemsep \parsep}%
   \belowdisplayskip \abovedisplayskip
}
\makeatother
% ---
% Espaçamentos entre linhas e parágrafos
% ---
 
% O tamanho do parágrafo é dado por:
\setlength{\parindent}{1.3cm}
 
% Controle do espaçamento entre um parágrafo e outro:
\setlength{\parskip}{0.2cm}  % tente também \onelineskip
 
% ---
% compila o indice
% ---
\makeindex
% ---
 
% ---
% GLOSSARIO
% ---
 
% ---
% entradas do glossario


% ---
% Exemplo de configurações do glossairo
\renewcommand*{\glsseeformat}[3][\seename]{\textit{#1}  
 \glsseelist{#2}}
% ---
              
                
% ----
% Início do documento
% ----
\begin{document}
 
% Retira espaço extra obsoleto entre as frases.
\frenchspacing
 
% ----------------------------------------------------------
% ELEMENTOS PRÉ-TEXTUAIS
% ----------------------------------------------------------
 
% ---
% Capa
% ---
\imprimircapa
% ---

% Lista de Figuras
% ---
\pdfbookmark[0]{Lista de Figuras}{lof}  % Cria um marcador para a lista de figuras no PDF
\listoffigures*
\cleardoublepage  % Insere uma nova página após a lista de figuras
% ---
% ---
% Lista de Tabelas
% ---
\pdfbookmark[0]{Lista de Tabelas}{lot}  % Cria um marcador para a lista de tabelas no PDF
\listoftables*
\cleardoublepage  % Insere uma nova página após a lista de tabelas
% ---
% ---
% inserir o sumario
% ---
\pdfbookmark[0]{\contentsname}{toc}
\tableofcontents*
\cleardoublepage
% ---
  
 
% ----------------------------------------------------------
% ELEMENTOS TEXTUAIS
% ----------------------------------------------------------
\textual
 
% ----------------------------------------------------------
% Introdução
% ----------------------------------------------------------
\chapter*[Introdução]{Introdução}
\addcontentsline{toc}{chapter}{Introdução}
 Apresentarei neste relatório uma análise descritiva dos dados de 2017 da Saeb (Sistema
 de Avaliação da Educação Básica). A Saeb é um conjunto de avaliações em larga escala que
 permite que o Inep faça um diagnóstico da educação brasileira e dos fatores que que a influênciam. 
 As avalições são feitas por meio de testes e questionários aplicados de 2 em 2
 anos. \\
 Neste relatório iremos avaliar as seguintes variáveis: \\
 REGIAO: Região de localização da escola do estudante (1-Norte, 2-Nordeste, 3-Sudeste, 4-Sul, 5-Centro-Oeste) \\
 LOCALIZACAO: Localização da escola do estudante(1-urbana ou 2-rural) \\
 COMPUTADOR: Tem computador em casa? (Não tem, Sim, um, Sim, dois, Sim, três, Sim, quatro ou mais) \\
 RACA\_COR: Raça/cor do estudante (A-Branca, B-Preta, C-Parda, D-Amarela, E-Indígena, F-Não quero declarar) \\
 NOTA\_LP: Proficiência em Língua Portuguesa transformada na escala única do SAEB, com média = 250, desvio = 50 (do SAEB/97) \\
 NOTA\_MT: Proficiência do aluno em Matemática transformada na escala única do SAEB, com média = 250, desvio = 50 (do SAEB/97) \\


 
% ---
% Capitulo que faz uso de elementos do glossario
% ---
\chapter{Análise Descritiva dos dados da Saeb}
\clearpage
% ---
\section{Variáveis qualitativas}
%-------------------------
% Raça-cor dos alunos
%-------------------------
\subsection{Raça/cor}
 \begin{figure}[h!]
    \centering
    \includegraphics[width=0.5\textwidth]{Figures/raca.png}
    \caption{Números de alunos por raça/cor}
    \label{fig:exemplo}
\end{figure}
\begin{table}[h!]
   \centering
\caption{Tabela com o percentual de participação por raça/cor autodeclarada.}
      \begin{tabular}{ccc}
      \hline
      Raça/cor           & \multicolumn{1}{l}{Frequência} & \multicolumn{1}{l}{Frequência(\%)} \\
      \hline
      \hline
      Parda              & 446.0                          & 44.6                               \\
      \hline
      Branca             & 299.0                          & 29.9                               \\
      \hline
      Não quero declarar & 108.0                          & 10.8                               \\
      \hline
      Preta              & 100.0                          & 10.0                               \\
      \hline
      Amarela            & 31.0                           & 3.1                                \\
      \hline
      Indígena           & 16.0                           & 1.6                                \\ \hline
\end{tabular} \\
{\footnotesize Fonte: Inep (Instituto Nacional de Estudos e Pesquisas Educacionais Anísio Teixeira).}
\label{tab:Raça/cor}
\end{table}
\vspace*{1cm}
\addcontentsline{toc}{subsection}{Raça/cor}
Podemos observar, por meio da tabela e do gráfico acima, que a maioria, 44,6\%, dos alunos
se autodeclaram pardos. O que coincide com o censo, no qual a maioria dos brasileiros se autodeclaram pardos.
\clearpage
%-------------------------
% Localização das escolas dos alunos
%-------------------------
\subsection{Localização}
 \begin{figure}[h!]
    \centering
    \includegraphics[width=0.5\textwidth]{Figures/localizacao.png}
    \caption{Localização das escolas onde os alunos realizaram as provas}
    \label{fig:exemplo}
\end{figure}
\begin{table}[h!]
   \centering
\caption{Quantidade de alunos por localização das escolas.}
\begin{tabular}{lcc}
\hline
Localização & \multicolumn{1}{l}{Frequência} & \multicolumn{1}{l}{Frequência(\%)} \\ \hline
\hline
Urbana      & 883.0                          & 88.3                               \\ \hline
Rural       & 117.0                          & 11.7                               \\ \hline
\end{tabular} \\
\footnotesize Fonte: Inep (Instituto Nacional de Estudos e Pesquisas Educacionais Anísio Teixeira).
\end{table}
\addcontentsline{toc}{subsection}{Localização}
A maioria absoluta (88,3\%) das escolas estão localizadas em zonas urbanas. O que era
esperado dado que a maioria da população brasileira vive em zonas urbanas.
\clearpage
%-------------------------
% Categoria administrativa das escolas
%-------------------------
\subsection{Categoria administrativa da escola dos alunos}
 \begin{figure}[h!]
    \centering
    \includegraphics[width=0.5\textwidth]{Figures/dependencia.png}
    \caption{Dependência administrativa das escolas dos alunos}
    \label{fig:exemplo}
\end{figure}
\begin{table}[h!]
   \centering
\caption{Dependência administrativa por alunos.}
\begin{tabular}{lcc}
\hline
Dependência & \multicolumn{1}{l}{Frequência} & \multicolumn{1}{l}{Frequência(\%)} \\ \hline
\hline
Estadual    & 526.0                          & 52.6                               \\ \hline
Municipal   & 471.0                          & 47.1                               \\ \hline
Federal     & 3.0                            & 0.3                                \\ \hline
\end{tabular} \\
\footnotesize Fonte: Inep (Instituto Nacional de Estudos e Pesquisas Educacionais Anísio Teixeira).
\end{table}
\addcontentsline{toc}{subsection}{Localização}
Pelo fato de existir um menor número de escolas federais no país, isso aparece nos dados,
já que a quantidade de escolas federais da amostra é muito pequeno, 0,3\% em comparaçao
com as escolas municipais (47,1\%) e estaduais (52,6\%) que são a maioria das escolas públicas. \\
A participação das escolas particulares no Saeb é facultativa. Portanto ou, nenhuma participou ou,
o número de escolas que participarão foi tão ínfimo que não apareceram na amostra.
\clearpage
%-------------------------
% Região das escolas
%-------------------------
\subsection{Região}
 \begin{figure}[h!]
    \centering
    \includegraphics[width=0.5\textwidth]{Figures/regiao.png}
    \caption{Região onde foram realizadas as provas}
    \label{fig:exemplo}
\end{figure}
% Please add the following required packages to your document preamble:
% \usepackage[normalem]{ulem}
% \useunder{\uline}{\ul}{}
\begin{table}[h!]
   \centering
\caption{Percentual de participação por região}
\begin{tabular}{lcc}
\hline
\text{Região} & \multicolumn{1}{l}{\text{Frequência}} & \multicolumn{1}{l}{\text{Frequência(\%)}} \\ \hline
\hline
Sudeste         & 347.0                                   & 34.7                                        \\ \hline
Nordeste        & 326.0                                   & 32.6                                        \\ \hline
Sul             & 138.0                                   & 13.8                                        \\ \hline
Norte           & 110.0                                   & 11.0                                        \\ \hline
Centro-Oeste    & 79.0                                    & 7.9                                         \\ \hline
\end{tabular} \\
\footnotesize Fonte: Inep (Instituto Nacional de Estudos e Pesquisas Educacionais Anísio Teixeira).
\end{table}
\addcontentsline{toc}{subsection}{Região}
Como podemos observar no gráfico a maior parte das escolas são das regiões Sudeste e Nordeste,
o que era esperado já que são as duas regiões mais populosas do Brasil.
\clearpage
%-----------------------------------
% Se o aluno possui computador em casa
%-----------------------------------
\subsection{Tem computador em casa?}
 \begin{figure}[h!]
    \centering
    \includegraphics[width=0.5\textwidth]{Figures/computador.png}
    \caption{Se o aluno possui computador em casa}
    \label{fig:exemplo}
\end{figure}
\begin{table}[h!]
   \centering
\caption{Quantos computadores o aluno possui em casa.}
\begin{tabular}{lcc}
\hline
Possui computador em casa? & \multicolumn{1}{l}{Frequência} & \multicolumn{1}{l}{Frequência(\%)} \\ \hline
\hline
Não tem                    & 436.0                          & 43.6                               \\ \hline
Sim, um                    & 414.0                          & 41.4                               \\ \hline
Sim, dois                  & 94.0                           & 9.4                                \\ \hline
Sim, três                  & 28.0                           & 2.8                                \\ \hline
Não respondeu              & 16.0                           & 1.6                                \\ \hline
Sim, quatro ou mais        & 11.0                           & 1.1                                \\ \hline
Não sei                    & 1.0                            & 0.1                                \\ \hline
\end{tabular} \\
\footnotesize Fonte: Inep (Instituto Nacional de Estudos e Pesquisas Educacionais Anísio Teixeira).
\end{table}
\addcontentsline{toc}{subsection}{Tem computador em casas?}
De todos os alunos que fizeram a avaliação 43,6\% não possuem computador em casa, enquanto 41,4\% possuem somente um computador
na sua residência. Isso nos mostra como a desigualdade de renda afeta o ensino, pois o computador
é, nos dias de hoje, uma ferramenta essencial para o aprensisado.
\clearpage
%------------------------
% Variáveia Quantitativas
%------------------------
\clearpage
%------------------------
% Notas matemática
%------------------------
\section{Variáveis Quantitativas}
\subsection{Nota de matemática}
 \begin{figure}[h!]
    \centering
    \includegraphics[width=0.5\textwidth]{Figures/histmat.png}
    \caption{Histograma das notas de matemática dos alunos do Saeb}
    \label{fig:exemplo}
\end{figure}
\begin{table}[h!]
   \centering
\caption{Notas de matemática por alunos separadas por classes.}
\begin{tabular}{lcc}
\hline
Notas de Matemática & \multicolumn{1}{l}{Frequência} & \multicolumn{1}{l}{Frequência} \\ \hline
\hline
260 - 285        & 200.0                          & 20.0                           \\ \hline
235 - 260       & 190.0                          & 19.0                           \\ \hline
210 - 235        & 166.0                          & 16.6                           \\ \hline
285 - 310        & 126.0                          & 12.6                           \\ \hline
185 - 210        & 116.0                          & 11.6                           \\ \hline
160 - 185        & 77.0                           & 7.7                            \\ \hline
310 - 335        & 65.0                           & 6.5                            \\ \hline
335 - 360        & 25.0                           & 2.5                            \\ \hline
135 - 160        & 24.0                           & 2.4                            \\ \hline
360 - 385        & 9.0                            & 0.9                            \\ \hline
385 - 410        & 2.0                            & 0.2                            \\ \hline
\end{tabular} \\
\footnotesize Fonte: Inep (Instituto Nacional de Estudos e Pesquisas Educacionais Anísio Teixeira).
\end{table}
\addcontentsline{toc}{subsection}{Notas de matemática}
Podemos observar ao analisarmos o histograma que as notas de matemática dos alunos seguem
uma distribuição normal.
\clearpage

 \begin{figure}[h!]
    \centering
    \includegraphics[width=0.5\textwidth]{Figures/boxmat.png}
    \caption{Boxplot das notas de matemática}
    \label{fig:exemplo}
\end{figure}
\begin{table}[h!]
   \centering
   \caption{Estatísticas das notas de matemática}
\begin{tabular}{|l|c|}
\hline
Estatísticas & \multicolumn{1}{l|}{Notas de Matemática} \\ \hline
Curtose      & 2,71                                     \\ \hline
Mediana      & 250,59                                   \\ \hline
Média        & 248,94                                   \\ \hline
Máximo       & 409,66                                   \\ \hline
Mínimo       & 135,46                                   \\ \hline
Assimetria   & 0,061                                    \\ \hline
\end{tabular}
\end{table}
\addcontentsline{toc}{subsection}{Boxplot e estatística de matemática}
Podemos observar ao anarlisarmos o boxplot que as notas de matemática tem uma variabilidade
baixa e com poucos outliers, somente dois no caso. \\
Podemos observar pelas estatísticas das notas de matemática que a mediana e a média são
praticamente iguais, dado que a distribuição das notas é simétrica. \\
A curtose é quase igual a 3 e a assimetria é muito pequena. Portanto podemos considerar
a distribuição simétrica.
\clearpage

\subsection{Nota de Português}
 \begin{figure}[h!]
    \centering
    \includegraphics[width=0.5\textwidth]{Figures/histport.png}
    \caption{Histograma das notas de português dos alunos do Saeb}
    \label{fig:exemplo}
\end{figure}
\begin{table}[h!]
   \centering
\caption{Notas de português por alunos separadas por classes.}
\begin{tabular}{lcc}
\hline
Notas     & \multicolumn{1}{l}{Frequência} & \multicolumn{1}{l}{Frequência(\%)} \\ \hline
\hline
255 - 280 & 213.0                          & 21.3                               \\ \hline
230 - 255 & 184.0                          & 18.4                               \\ \hline
205 - 230 & 158.0                          & 15.8                               \\ \hline
280 - 305 & 132.0                          & 13.2                               \\ \hline
180 - 205 & 117.0                          & 11.7                               \\ \hline
305 - 330 & 74.0                           & 7.4                                \\ \hline
155 - 180 & 57.0                           & 5.7                                \\ \hline
330 - 355 & 33.0                           & 3.3                                \\ \hline
130 - 155 & 20.0                           & 2.0                                \\ \hline
355 - 380 & 10.0                           & 1.0                                \\ \hline
\end{tabular} \\
\footnotesize Fonte: Inep (Instituto Nacional de Estudos e Pesquisas Educacionais Anísio Teixeira).
\end{table}
\addcontentsline{toc}{subsection}{Notas de português}
O histograma nos mostra que as notas de português, tal qual as de matemática, 
também seguem um distribuição normal. Com a diferença que tivemos um número um pouco
maior de notas acimas de 350 na prova de matemática.
\clearpage
 \begin{figure}[h!]
    \centering
    \includegraphics[width=0.5\textwidth]{Figures/boxport.png}
    \caption{Boxplot das notas de português}
    \label{fig:exemplo}
\end{figure}
\begin{table}[h!]
   \centering
   \caption{Estatísticas das notas de português}
\begin{tabular}{|l|c|}
\hline
Estatísticas & \multicolumn{1}{l|}{Notas de Português} \\ \hline
Curtose      & 2,56                                    \\ \hline
Mediana      & 256,81                                  \\ \hline
Média        & 253,00                                  \\ \hline
Máximo       & 374,13                                  \\ \hline
Mínimo       & 130,58                                  \\ \hline
Assimetria   & -0,065                                  \\ \hline
\end{tabular}
\end{table}
\addcontentsline{toc}{subsection}{Boxplot e estatísticas de português}
Tal qual o boxplot de matemática, o de português também possui pouca variabilidade,
mas diferente do de matemática as notas de português não possuem outliers. \\
A mediana e a média estão bastante próximas.
A curtose é um pouco menor que a curtose das notas de matématica, mas mesmo assim muito
próxima de 3 e a assimetria negativa (assimetria a esquerda), diferente da de matématica que era positiva, 
mas também muito pequena. \\
Portanto podemos considerar que as notas de português seguem
uma distribuição simétrica.
\clearpage

\subsection{Comparativo entre as notas de português e matemática}
 \begin{figure}[h!]
    \centering
    \includegraphics[width=0.5\textwidth]{Figures/hist_comparacao.png}
    \caption{Histograma comparando as notas de português e matemática}
    \label{fig:exemplo}
\end{figure}
\addcontentsline{toc}{subsection}{Comparativo entre as notas}
Para finalizar, temos um histograma comparando as duas distribuições. Podemos observar
que as distribuições são muito parecidas, o que difere as duas são as pequenas assimetrias,
positiva no caso das notas de matématica e negativa no caso das notas de português.
\clearpage


% ----------------------------------------------------------
% Capitulo com exemplos de comandos inseridos de arquivo externo 
% ----------------------------------------------------------

\include{abntex2-modelo-include-comandos}
 
 
% ----------------------------------------------------------
% ELEMENTOS PÓS-TEXTUAIS
% ----------------------------------------------------------
\postextual
 
 
% ----------------------------------------------------------
% Referências bibliográficas
% ----------------------------------------------------------
\bibliography{abntex2-modelo-references}
 
  
 
% ----------------------------------------------------------
% Apêndices
% ----------------------------------------------------------
 
% ----------------------------------------------------------
% Anexos
% ----------------------------------------------------------
 
%---------------------------------------------------------------------
% INDICE REMISSIVO
%---------------------------------------------------------------------
 
\end{document}
